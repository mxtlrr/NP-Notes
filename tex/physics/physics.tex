% chktex-file 44
% chktex-file 8
% chktex-file 3

\documentclass{article}

\usepackage{mhchem}
\usepackage{multicol}
\usepackage[margin=50px]{geometry}
\usepackage{enumitem}
\usepackage{titlesec}

\setlist[enumerate]{label*=\arabic*.}


\usepackage{hyperref}
\hypersetup{
  colorlinks=true,
  urlcolor=blue,
  pdftitle={Nuclear Physics},
}

\title{Nuclear Physics}
\author{}
\date{2023}

\begin{document}
  \maketitle
  \tableofcontents

  \newpage
  
  \newpage
  \begin{multicols*}{2}
    \section{What is Nuclear Physics?}
    Nuclear physics is the field of study that concerns atomic nuclei,
    their constituents, and their interactions with other nuclei. It
    should not be confused with \textbf{atomic physics} which concerns
    the entire atom.

    \section{Isotopic Notation}
    Isotope notation is a way of writing isotopes in reactions. It consists
    of 3 parts, $A$, $Z$, and $X$:

    \[
      \ce{^A_Z X}
    \]

    Where $A$ is the mass number ($A = Z + N$, $N$ is the amount of neutrons),
    $Z$ is the atomic number -- the amount of protons, and finally $X$ is
    the chemical symbol. For example $\ce{^6_3 Li}$ corresponds to Lithium-$6$,
    $Z=3, N=3$ thus $A=6$. 
    \section{Nuclear Reactions}
    Nuclear reactions are a process in which 2 nuclei collide to produce 1 or 
    more nuclides. It is a chemical reaction, but on an atomic scale.

    \subsection{Nuclear Equations}
    The notation for a nuclear equation is almost the same as a chemical reaction,
    however it contains isotopic notation:

    \[
      \ce{^6_3Li + ^2_1H -> ^4_2He + \ldots}
    \]

    To figure out what goes in place of $\ldots$ we have to balance. You can think
    of it like a system of equations: 
    \[6+2 = 4+x\]
    \[3+1 = 2+y\]

    We can solve for $x$ and $y$, giving us $x=4$ and $y=2$, which correspond to
    $\ce{^4_2He}$. Therefore we can write this equation in two ways:

    \[ \ce{^6_3Li + ^2_1H -> ^4_2He + ^4_2He} \]
    \[ \ce{^6_3Li + ^2_1H -> 2 ^4_2He} \]

    \subsection{Short Form Equations}
    In most situations, people use short form notation to describe nuclear reactions.
    Essentially:

    \[ A(b,c)D \implies \ce{A + b -> c + D}\]

    \section{Energy}
    \subsection{The Definition of Electronvolts}
    Electronvolts are how we measure energy in nuclear reaction. By definition,
    a electronvolt is the measure of an amount of kinetic energy gained by
    an electron accelerating to an electric potential different of one volt
    in a vacuum. It is equal to $1.602176634 \times 10^{-19}$ J. $1 \text{ MeV}$
    is equal to $1 \times 10^6$ eV.

    \subsection{Energy Conservation in Nuclear Reactions}
    In this subsection, we will be using the equation shown in section 3.1.
    Kinetic energy may be released (\textbf{exothermic}) or supplied
    (\textbf{endothermic}) for the reaction to take place. To calculate how
    much kinetic energy is needed or released, we can reference a table\footnote{this is found in `isotopes.txt' of the same directory.}
    of very accurate particle rest masses.\footnote{you can also go to \href{https://physics.nist.gov/cgi-bin/Compositions/stand_alone.pl?ele=&all=all&ascii=ascii&isotype=all}{this url}}\\

    For example, $\ce{^6Li}$ has a rest mass of $6.014 u$, $\ce{^2H}$ has
    $2.014 u$, and $\ce{^4He}$ has $4.0026 u$. The sum of our rest mass is
    $8.029 u$. The product's rest mass is $2(4.0026) = 8.0052$. Notice how
    there's a difference between $8.029$ and $8.0052$. That is the kinetic
    energy -- $0.0238$. To find out how much energy is required in this
    reaction, we must do some math:

    \begin{equation*}
      \begin{split}
        1 \text{u c}^2 = (1.66054 \times 10^{-27} \text{ kg}) \times (2.99792 \times 10^{8} \text{ m/s})^2\\
        = (1.49242 \times 10^{-10} \text{ kg (m/s)}^2)\\
        = 1.49242 \times 10^{-10} \text{ J} \times
        \frac{1 \text{ MeV}}{1.60218 \times 10^{-13} \text{ J}}\\
        = 931.49 \text{ MeV}
      \end{split}
    \end{equation*}

    Thus, 1 atomic mass unit is $931.49$ MeV. \textbf{This is constant along
    every single reaction}. Finally, we can do $0.0238 \times 931$ to find the
    energy required, which gives us $22.2$ MeV.\\

    This value is known as the Q-Value. If this value is positive, the reaction
    is exothermic. If it is negative, it is endothermic. In nuclear equations,
    it is written at the end of equation, for example:

    \[
      \ce{^6_3Li + ^2_1H -> 2 ^4_2He + 22.2 MeV}
    \]

    \subsection{Subatomic Particles' Mass}
    In the calculation of Energy/the Q-Value, some reactions have subatomic particles,
    for example:
    \[
      \ce{^{249}_{98}Cf + ^{48}_{20}Ca -> ^{294}_{118}Og + 3 n}
    \]

    The below table shows subatomic particles and their rest masses.\\

    \begin{tabular}{|c|c|c|}
      \hline
      \textbf{Particle} & \textbf{Mass (u)} & \textbf{Notation} \\
      \hline
      Alpha   & $4.002$  & $\ce{^4_2He}$     \\
      Beta    & $0.0005$ & $\ce{^0_{-1}e}$   \\
      Gamma   & $0$      & $\ce{^0_0\gamma}$ \\
      \hline
      Proton  & $1.007$  & N/A \\
      Neutron & $1.003$  & N/A \\
      \hline
    \end{tabular}

    \section{Particles}
    There are various particles in nuclear physics/nuclear chemistry that are
    important, as they contribute to decay.\\

    \begin{tabular}{|c|c|c|}
      \hline
      Particle & Notated As & Changes to A and Z \\
      \hline
      $\alpha$ & $\ce{^4_2He}$ & $A+4$, $Z-2$ \\
      $\beta$ & $\ce{^0_-1e}$ & $A+0$, $Z+1$ \\
      $\gamma$ & $\ce{^0_0\gamma}$ & $A+0$, $Z-0$\\
       &  &   \\
      Positron & $\ce{^0_{+1}e}$ & $A+0$, $Z-1$\\
      EC & $\ce{^0_-1e}$ & $A+0$, $Z-1$\\
      \hline
    \end{tabular}

    \subsection{The Case for the Gamma Particle}
    You may have noticed that the $\gamma$ particle does not change $A$ or $Z$.
    Why is this? This is due to something called an \textbf{excited state}.
    
    \subsubsection{Excited State}
    To put it simply, in an excited state, $1$ or more sub-atomic particles in
    the nucleus occupy a nuclear orbital of higher energy than an available
    nuclear orbital.\\
    
    These nuclear isomers have higher decay rates, due to the decay being
    `forbidden' from the large change in nuclear spin needed to emit a
    gamma ray. For example $\ce{^{99m}Tc}$ has a spin of $1/2$ and must
    $\gamma$-decay to $\ce{^{99}Tc}$ with a spin of $9/2$.

    \section{Decay Methods}
    \subsection{Alpha Decay}
    Alpha decay is a type of radioactive decay where a radioactive nucleus
    emits an alpha particle (which is equivalent to $\ce{^4_2He}$). It most
    commonly occurs in the heaviest nuclides. That, along with spontaneous
    fission.

    \subsection{Beta Decay}
    Beta decay is a type of decay when an atomic nucleus releases a beta
    particle. Beta particles are essentially fast energetic electron or
    positron, transforming it into an isobar of the original nuclide.\\

    Isobars are atoms of different elements that have the same number of
    nucleons. They differ in atomic number but have the same $A$ value.
    For example, a series of isobars is shown below:

    \[
      \ce{^{40}S}\text{, }\ce{^{40}Cl}\text{, }\ce{^{40}Ar}
    \]

    \subsection{Double Beta Decay}
    Double Beta Decay is a rare (only 35 naturally occuring isotopes which
    undergo this form of decay) form of beta decay, in which two neutrons
    are simultaneously transformed into protons:

    \[
      \ce{2 ^{1}_{0}n -> 2 ^{1}_{0} p}
    \]

    There are two types of double beta decay: ordinary and neutrinoless.

    \subsubsection{Ordinary}
    Typical $2\beta\beta$ consists of two neutrons in the nucleus being converted,
    and two electrons and two electron antineutrinos, $\ce{\nu^{-}_{e}}$,
    are released. It can be thought of as two simultaneous $\beta^{-}$ decays.\\

    As mentioned earlier, there are only 35 natural isotopes that undergo
    this form of decay. The lightest of which, is $\ce{^{48}Ca}$ and the heaviest
    is $\ce{^{238}U}$. These isotopes tend to have incredibly long half-lives.\\

    The below table shows examples of these isotopes followed by their half-lives.\\
    
    \begin{tabular}{|c|c|}
      \hline
      \textbf{Isotope} & \textbf{Half-life (years)} \\
      \hline
      $\ce{^{48}Ca}$ & $6.4 \times 10^{19}$ \\
      $\ce{^{76}Ge}$ & $1.926 \times 10^{21}$ \\
      $\ce{^{82}Se}$ & $0.97 \times 10^{20}$ \\
      $\ce{^{96}Zr}$ & $20 \times 10^{18}$ \\
      \hline
    \end{tabular}

    The formula for the change of $A$ and $Z$ is shown below:
    \[
      \left(A, Z\right) \rightarrow \left(A, Z + 2\right) + 2e^{-}
    \]

    \subsubsection{Neutrinoless $2\beta\beta$ ($0\nu\beta\beta$)}
    This form of $2\beta\beta$ has not been discovered yet and is purely theoretical.
      

    \subsection{Isomeric Decay}
    This method of decay is more known as isomeric transition. It can occur
    in two methods:
    \begin{enumerate}
      \item $\gamma$ ray
      \item Internal Conversion
    \end{enumerate}
    \subsubsection{Gamma}
    Gamma decay, also known as gamma radiation, is a penetrating form of
    electromagnetic radiation arising from radioactive decay of nuclei. These
    forms of rays are used for the removal of cancerous cells.
    
    \subsubsection{Internal Conversion (IT)}
    Essentially, this means that the energy the nucleus has to eject an electron.
    The electron is a `high energy' one, and is emitted from the \textbf{atom},
    not the nucleus. For this specific reason, the particles are not called
    beta particles.\\

    For example, take the decay scheme of $\ce{^{203}Hg}$. $\ce{^{203}Hg}$ produces
    a continous beta spectrum (max energy is $214$ keV). This leads to an excited
    state of $\ce{^{203}Tl}$. The excited state decays extremely fast and releases
    a gamma quantum of $279$ keV.

    \begin{equation*}
      \begin{split}
        \ce{^{203}_{80}Hg ->[\beta^{-}][46.95 d] ^{203m}_{81}Tl} \\
        \ce{^{203m}_{81}Tl ->[IT][7.7 \mu{s}] ^{203}_{81}Tl} \\
      \end{split}
    \end{equation*}
    \subsection{Neutron Emission}
    Neutron emission is a form of nuclear decay in which one or more neutrons.
    It occurs in most nuclides that contain more neutrons than protons. For
    example, $\ce{^{4}H}$ has $3$ neutrons and $1$ proton. It decays as such
    \[
      \ce{^{4}_{1}H ->[139 ys] ^{1}_{0}n + ^{3}_{1}H}
    \]
    
    \subsection{Proton Emission}
    Proton emission is when a nuclide ejects a proton. It can occur from
    the ground state / low-lying isomer of very proton-rich nuclei. In
    this case, the process is similar to that of alpha decay.\\

    An example of this occuring is $\ce{^{45}Fe}$, which has a $30\%$ chance
    of undergoing beta decay, and $70\%$ chance of undergoing proton emission.
    \begin{equation*}
      \begin{split}
        \ce{^{45}_{26}Fe ->[\beta^{+}] ^{45}_{25}Mn + ^{1}_{0}n} \\
        \ce{^{45}_{26}Fe ->[2p] ^{43}_{24}Cr + + 2 ^{1}_{1}p}
      \end{split}
    \end{equation*}

    \subsection{Electron Capture}
    Electron capture is a process in which a proton-rich nucleus of an electically
    neutral atom absorbs an inner atomic an inner atomic electron.\footnote{usually from the K or L electron shells}
    This changes a proton to a neutron, and releases a neutrino:
    \[
      \ce{p + e- \to n + \nu_e}
    \]

    Or, when written as a nuclear equation:
    \[
      \ce{^{0}_{-1}e + ^{1}_{1}p -> ^{1}_{0}n + ^{0}_{0}\nu_{e}}
    \]
    The table below shows common isotopes that decay solely via electron capture
    and their half-lives: \\

    \begin{tabular}{|c|c|}
      \hline
      \textbf{Isotope} & \textbf{Half Life} \\
      \hline
      $\ce{^{7}_{4}Be}$ & $53.28$ d \\
      $\ce{^{37}_{18}Ar}$ & $35.0$ d \\
      $\ce{^{44}_{22}Ti}$ & $60$ y \\
      \hline
    \end{tabular}

    \subsection{Double Electron Capture ($\varepsilon\varepsilon$)}
    This form of nuclear decay is \textit{INCREDIBLY} rare. Only 3 nuclides,
    $\ce{^{78}Kr}$, $\ce{^{130}Ba}$ and $\ce{^{124}Xe}$, undergo this form of decay.
    A reason for this rarity is that the half lives for this mode are extremely
    long.\\

    \begin{tabular}{|c|c|}
      \hline
      \textbf{Isotope} & \textbf{Half Life (years)} \\
      \hline
      $\ce{^{78}_{36}Kr}$ & $9.2 \times 10^{21}$ \\
      $\ce{^{130}_{56}Ba}$ & $1.6(\pm{1.1})\times 10^{21}$ \\
      $\ce{^{124}_{54}Xe}$ & $1.8 \times 10^{22}$ \\
      \hline
    \end{tabular}

    An example of how one of these isotopes decays is shown below.
    \[
      \ce{^{130}_{56}Ba + 2 e- ->[\varepsilon\varepsilon][1.6 \times 10^{21}] ^{130}_{54}Xe + 2 \nu_{e}}
    \]

    \subsection{Still-Unobserved Decay}
    The continual imporvement of sensitivity will allow the discovery of mild
    instability of isotopes that are considered stable today. A modern example is as
    such: $\ce{^{209}Bi}$ was thought to not be radioactive until $2003$. In $2003$,
    researchers found that it \textit{does decay} with a half life of $2.01 \times 10^{19}$
    via $\alpha$ decay.\\

    There are some isotopes that are predicted to be unstable, but haven't been observed
    to do this activity. These are known as \textbf{observationally stable}. There are
    $161$ of these isotopes, $45$ of which have been observed in detail with no sign of
    decay. The lightest in any case is $\ce{^{36}Ar}$.

    \section{Decay Chain}
    A decay chain is a series of radioactive decays of different decay products
    as a series of transformations. Decay stages are referred to by their relationship
    to previous/subsequent stages.\\

    The most well known decay chain is the $4n$ chain of $\ce{^{232}Th}$. It is known
    as the $4n$ chain due to the majority of decay modes being $\alpha$. The beginning
    of the decay chain is shown below:

    \begin{equation*}
      \begin{split}
        \ce{^{232}_{90}Th ->[\alpha][14\cdot{10^{9}} y] ^{228}_{88}Ra
          ->[\beta{-}][5.7 y] ^{228}_{89}Ac ->[\beta-][6.1 hr] ^{228}_{90}Th
          ->[\alpha][1.9 y] ^{224}_{88}Ra}
      \end{split}
    \end{equation*}

    This chain ends at $\ce{^{208}Pb}$, and has $11$ steps, it splits at the
    10th step, depending
    on how $\ce{^{212}Bi}$ decays. If it decays via $\alpha$ ($35.94\%$ chance),
    it converts to $\ce{^{212}Po}$ and then $\ce{^{208}Pb}$. If it decays via
    $\beta{^{-}}$ ($64.05\%$ chance) it decays to Thallium and then Lead:

    \begin{equation*}
      \begin{split}
        \ce{^{208}_{82}Pb <-[\beta{-}][3.1 min] ^{208}_{31}Tl <-[\alpha] ^{212}_{83}Bi
            ->[\beta{-}] ^{212}_{84}Po ->[\alpha][3 \times 10^{-7}]
            ^{208}_{82}Pb}
      \end{split}
    \end{equation*}

    \subsection{Parent Isotope}
    A parent isotope is an isotope that starts a decay chain. For example, uranium
    (Z $=92$) decays into thorium (Z $=90$). The time it takes, for the parent isotope
    to decay to the daughter isotope can vary. The decay of each atom, however, occurs
    spontaneously.

    \subsection{Daughter Isotope}
    A daughter isotope is any isotope that is not the parent isotope. In the $4n$ decay
    chain, every single isotope that is not $\ce{^{232}_{90}Th}$ is a daughter isotope. 

    \section{What is Spin?}
    Since an atomic nucleus is a particle of subatomic scale, it can be
    described with a set of `quantum properties'. One of these, is `nuclear
    spin', which is related to the sensitivity of the nucleus to the effects
    of an external magnetic field.\\

    With some oversimplification, a nucleus with a non-zero spin value
    can be viewed a magnet, sensitive to the presence of an external magnetic
    field.\\

    \begin{tabular}{|c|c|}
      \hline
      Spin value & Example Isotopes \\
      \hline
      $0$ & $\ce{^{12}C}$, $\ce{^{16}O}$ \\
      $1/2$ & $\ce{^{1}H}$, $\ce{^{13}C}$ \\
      $1$ & $\ce{^{2}H}$ \\
      \hline
    \end{tabular}

    \subsection{Determining Spin}
    There are $3$ rules you must take into account. They are:
    \begin{enumerate}
      \item If the number of neutrons and protons are both even, there is no spin.
      \item If $N+P$ is odd (where $N$ is number of neutrons and $P$ is number of protons),
      there is a half-integer spin ($1/2$, $3/2$, $5/2$)
      \item If the number of neutrons and protons are both odd, then the spin is an integer
      (e.g. $1$, $2$, $3$).
    \end{enumerate}

    \section{Transmutation of Elements}
    Transmutation is the conversion of one nuclide to another. This can happen in two ways:
    radioactive decay and reaction between two nuclides. We will cover the latter. The first
    man-made transmutation was in 1919 by Ernest Rutherford:
    \[
      \ce{^{14}_7N + ^4_2{He} -> ^{17}_8O + ^1_1H}
    \]

    In the case of plutonium, it is generally formed via $\beta^{-}$ decay,
    starting from $\ce{^{238}U}$:
    \[
      \ce{^{238}_{92}U + ^1_0n -> ^{239}_{92}U ->[\beta^{-}] ^{239}_{93}Np ->[\beta^{-}] ^{239}_{94}Pu}
    \]

    \subsection{Preparing Transuranium Elements}
    \textbf{Note: This table goes from Z=95 to Z=100}\\

    \begin{tabular}{|c|c|}
      \hline
      Z number & Reaction \\
      \hline
      $95$ & $\ce{^{239}_{94}Pu + ^1_0n -> ^{240}_{95}Am + ^0_{-1}e}$         \\
      $96$ & $\ce{^{239}_{94}Pu + ^4_2He -> ^{242}_{96}Cm + ^1_0n}$           \\
      $97$ & $\ce{^{241}_{95}Am + ^4_2He -> ^{243}_{97}Bk + 2 ^1_0n}$         \\
      $98$ & $\ce{^{242}_{96}Cm + ^4_2He -> ^{245}_{98}Cf + ^1_0n}$           \\
      $99$ & $\ce{^{238}_{92}U + 15 ^1_0n -> ^{253}_{99}Es + 7 ^0_{-1}e}$     \\
      $100$ & $\ce{^{204}_{82}Pb + ^{40}_{18}Ar -> ^{241}_{100}Fm + 3 ^1_0n}$ \\
      \hline
    \end{tabular}

    \section{Fusion vs Fission}
    \subsection{Nuclear Fission}
    Nuclear fission is a reaction when the nucleus of an atom splits into 2
    or more smaller nuclei. This process generally releases gamma photons,
    as well as a large amount of energy. There are two ways that fission can
    take place.

    \subsubsection{Spontaneous Fission (SF)}
    Spontaneous Fission is the spontaneous breakdown of one nucleus into
    smaller nuclei. It generally only occurs when $Z > 100$. There, however,
    is a formula that can approximately predict spontaneous fission that can
    occur in a time short enough to be observed. It is:

    \[
      \frac{Z^2}{A} \approx 47
    \]

    Currently, no known isotope other than $\ce{^{294}Og}$ ($47.36$)
    reaches a value of 47. This is due to this model \textbf{not being accurate}
    for the heaviest known nuclides.

    Even then, $\ce{^{294}Og}$ does not \textbf{only undergo} SF, as it also
    goes through $\alpha$ decay:

    \[
      \ce{^{294}_{118}Og ->[\alpha][700 \mu{s}] ^{290}_{116}Lv}
    \]
    \subsection{Nuclear Fusion}
    Nuclear fusion is a reaction in which two or more atomic nuclei, usually
    hydrogen variants, are combined to form one or more atomic nuclei. This
    is the process that powers active stars.

    If a nuclear process that produces atomic nuclei lighter than $\ce{^{56}Fe}$
    or $\ce{^{62}Ni}$ will generally release energy.

    \section{Nucleosynthesis}
    Nucleosynthesis is the creation of elements by fusion reactions within
    stars. In our solar system, the sun goes through Proton-proton chain reaction.
    In chemical equations, it is written as:
    \begin{equation*}
      \begin{split}
        \ce{2 ^{1}_{1}H -> ^{2}_{1}H + ^0_{+1}e + \nu}      \\
        \ce{^{2}_{1}H + ^{1}_{1}H -> \gamma{} + ^{3}_{2}He} \\
        \ce{2 ^{3}_{2}He -> 2 ^{1}_{1}H + ^{4}_{2}He}       \\
      \end{split}
    \end{equation*}

    \subsection{S-Process}
    The s-process is a process that is responsible for the nucleosynthesis
    of approximately $50\%$ of the atomic nuclei heavier than iron. The process
    has a few steps:
    \begin{enumerate}
      \item A seed nucleus undergoes neutron capture to form an isotope with $A+1$.
      \begin{enumerate}
        \item If isotope is stable, a series of increases of mass can occur.
        \item If it is not, then the nuclei will undergo $\beta$ decay to produce
        a nucleus with $Z=Z+1$.
      \end{enumerate}
    \end{enumerate}

    This process is slow, hence the name `s-process' which means `slow process'.
    The reason it is slow is it allows the decay to occur before another neutron
    get captured.\\

    The process is always terminated at a cycle involving lead, bismuth and
    polonium. The nuclear reactions shown below are the main neutron source
    reactions:

    \begin{equation*}
      \begin{split}
        \ce{^{13}_{6}C + ^{4}_{2}He -> ^{16}_{8}O + n} \\
        \ce{^{22}_{10}Ne + ^{4}_{2}He -> ^{25}_{12}Mg + n}
      \end{split}
    \end{equation*}
    
    \subsection{R-Process}
    The R-Process, also known as the rapid neutron-capture process, is a set
    of reactions that is responsible for around $50\%$ of atomic nuclei heavier
    than iron.

    \section{Nuclear Decay}
    \subsection{Half-Lives}
    A half-life of a radioactive isotope is the amount of time it takes for
    half of a sample of a isotope to decay. The half-life of a specific
    isotope \textbf{is constant}.\\

    To find how much of a sample remains after an amount of time, we first
    need to find the rate constant $K$:
    \[
      K = \frac{\ln 2}{t_{1/2}}
    \]

    For example, for $\ce{^{131}I}$, $K=0.08664$. To find the amount left
    over after an amount of time:

    \[
      A_F = A_o \times e^{-kt}
    \]

    where, $t$ is the amount of time, $A_o$ is the initial amount. The units
    of $t$ should be the same for the half life.

    \subsubsection{Calculating Half-Life}
    The equation to calculate a half life is shown below:
    \[
      t_{1/2} = \frac{t}{ \log_{0.5}\left( \frac{N_t}{N_0} \right) }
    \]

    Where, $t$ is the amount of time, $N_t$ is the quantity of matter remaining,
    and $N_0$ is the initial quantity of the material. For example, let's find
    the half life of $\ce{^{191}Tl}$. Let's say we have 10 grams of this isotope
    and measure it after 40 minutes.

    \begin{equation*}
      \begin{split}
        t_{1/2} = \frac{t}{ \log_{0.5}\left( \frac{N_t}{N_0} \right) } \\
        = \frac{40}{ \log_{0.5}\left( \frac{2.5}{10} \right) } \\
        = \frac{40}{2} \\
        = 20.
      \end{split}
    \end{equation*}
    
    This checks out, as $\ce{^{191}Tl}$ \textit{is actually} 20 minutes!
    \subsection{The Decay Constant}
    There are no tables for looking up the decay constant of some isotope.
    However, there is a formula to find this out:

    \[
      \lambda = \frac{0.693}{T}
    \]

    Where, $T$ is the half-life of the nuclide. The decay constant is the
    instantaneous decay fraction - the fraction of the activity that decays
    in an infinitesimally small amount of time.\\

    The decay constant is immutable. This means that is not affected by time,
    space, pressure or anything like that.
    
    \subsection{The Decay Equation}
    On the bright side, the following few equations are useful because
    they tell us the number of atoms / activity of the radionuclide as
    a function of time:

    \begin{equation*}
      \begin{split}
        N_t = N_{0}e^{-\lambda{t}} \\
        A_t = A_{0}e^{-\lambda{t}} \\
      \end{split}
    \end{equation*}

    Where $N_t$ is the number of atoms at time $t$, $N_0$ is the number
    of atoms at time $0$. $A_t$ is the activity of a radionuclide at time $t$,
    and $A_0$ is the activity at time $0$. Finally $e = 2.7182\ldots$.\\

    Let's do an example: Let's find the activity of a 100 curie sample of
    $\ce{^{131}I}$ after decaying for $1.5$ minutes.

    \begin{equation*}
      \begin{split}
        A_t = A_{0}e^{-\lambda{t}} \\
        = 100e^{-0.0862 \times 15} \\
        = 100e^{-0.1293} \\
        = 100 \times 0.8787 \\
        \approx 88.
      \end{split}
    \end{equation*}

    \subsubsection{Calculating Decay Time}
    To figure out how much time ($t$) is required for a radionuclide to decay
    from an initial activity $A_0$ to a specified final activity ($A_t$), the
    following equation can be used:

    \[
      t = -\frac{\ln{\frac{A_t}{A_0}}}{\lambda}
    \]

    Another equation that might be useful is calculating the amount of half lives
    required ($N_h$) to reduce the original activity $A_0$, to it's desired value $A_t$:
    \[
      N_h = 1.443 \times \ln{\left( \frac{A_t}{A_0} \right)}
    \]

    \section{Curies, Becquerels, and Activity}
    A curie (Ci) and a becquerel (Bq) are both units that measure the amount a radioisotope
    decays. $1 \text{ Bq}$ is equal to 1 decay per second, and $1 \text{ Ci}$ is equal to
    $3.70 \times 10^{10} \text{ Bq}$. Since a curie is non-SI (not official), that means that


    \subsection{Calculating Activity}
    It can be calculated via the following formula:
    \[
      R = \lambda{N}\\
    \]

    $N$ is the amount of radioactive nuclei, and $\lambda$ is the decay constant, defined as
    \[
      \lambda = \frac{0.693}{t_{1/2}},
    \]

    where $t_{1/2}$ is the half-life. It should be noted that in the equation $R=\lambda{N}$,
    $\lambda$ must be in seconds, so if you have some half life that is not in seconds, you
    must convert it to seconds.

    To calculate $N$, we must figure out the molar mass. This is generally going to be $A$ (e.g.
    for $\ce{^{63}Cu}$, the molar mass is $63$). Then we multiply that by Avogrado's number:
    \[
      N = \frac{S}{M} \times \left(6.022 \times 10^{23}\right),
    \]

    where $S$ is the quantity of the sample in grams, and $M$ is the molar mass.

    \subsubsection{Example}
    Assume we have $0.56$ grams of $\ce{^{252}-Cf}$. What is it's activity in Curies? We know
    two things:
    \begin{enumerate}
      \item Molar mass of the isotope ($252$)
      \item The half life is $2.645$ years.
    \end{enumerate}

    So, let's do some math.
    \begin{equation*}
      \begin{split}
        \lambda = \frac{0.693}{83469852} = 8.302 \cdot 10^{-9} \\
        N = \frac{0.56}{252} \times \left(6.022 \times 10^{23}\right) \\
          = 0.002 \times \left(6.022 \times 10^{23}\right) \\
          = 1.338\times 10^{21}\\
        \\
        R = \lambda{N} = (8.302 \times 10^{-9}) \times (1.338 \times 10^{21}) \\
        = \boxed{1.11080 \times 10^{13}}
      \end{split}
    \end{equation*}

    Now that we have the decays per second, we should convert it into Curies:
    \begin{equation*}
      \begin{split}
        R_C = \frac{1.11080 \times 10^{13}}{3.7\times{10^{10}}} \\
        = \boxed{300.216 \text{ Ci}}
      \end{split}
    \end{equation*}
    \section{Isomers}
    A nuclear isomer is a metastable state of an atomic nucleus. The term
    `metastable' describes a ngod damn isuclei whose excited states have half-lives $100$
    to $1000$ times longer than the half-lives of the excited nuclear states
    that decay with a prompt half life.\\

    Metastable isomers of a particular isotope are designated with an `m',
    for example $\ce{^{58m1}_{27}Co}$. For isotopes that have more than one
    of these isomers, indices are placed after the definition ($m1$, $m2$,
    $m3$).\\

    The increased index correspond to the increasing to the levels of excitation
    energy stored in each of the isomeric states. For example, take the $3$
    metastable isomers of $\ce{^{178}Hf}$:\\

    \begin{tabular}{|c|c|c|}
      \hline
      Isomer & Excitation Energy (KeV) & Half life \\
      \hline
      $\ce{^{178m1}Hf}$ & 1147.42 & 4.02s \\
      $\ce{^{178m2}Hf}$ & \textbf{2445.69} & 31 y \\
      $\ce{^{178m3}Hf}$ & 2573.5 & 68 $\mu$s \\
      \hline
    \end{tabular}

    \subsection{What is excitation energy?}
    Exictation energy is the minimum amount of energy to promote a molecule
    from its ground state to an excited state.


    \section{Nuclear Force}
    The nuclear force is, in simple terms, what keeps nuclei together. It
    acts between the protons and neutrons (nucleons) of atoms, both affected
    almost identically.\\

    Now, since protons have a charage of $+1\text{ }e$, they (they meaning
    both nucleons) tend to push them apart. However, at such as short range
    (which is $0.8$ femtometers, or $0.8 \times 10^{-15}$ meters), the nuclear
    force can overcome this electrical force and keep the nucleons together.

    \section{Decay Technique}
    The decay technique is a technique used to synthesize various unstable chemical
    compounds, such as $\ce{XeCl4}$. The technique is relatively new, being developed
    in 1963.

    \subsection{Carbocation generation}
    In the basic method, a compound with the structure $\ce{(R1-R2-R3)C-T}$\footnote{T for tritium}
    is prepared. As tritium undergoes $\beta$ decay ($t_{1/2} = 12.32$y), it becomes
    a $\ce{^{3}He}$ ion:
    \[
      \ce{(R1-R2-R3)C-[^{3}He]}
    \]

    The helium atom always breaks away and forms a carbocation, $\ce{[(R1,R2,R3)C+]}$.
    In the case of tritiated methane, $\ce{CH3T}$, this forms a carbenium ($\ce{H3C+}$) ion.

    \subsection{Persistent Bound Structures}
    The carbon-helium bond breaks almost instantaneously, bonds of other elements to helium
    are much more stable. Take for example $\ce{TH}$. On decay, this forms the
    helium hydride ion, $\ce{[HeH]+}$.\\

    This method is a way of synthesizing cations that cannot be made in any other way. For example,
    For example, the perbromate ion was made via beta decay of selenium-83:
    \[
      \ce{^{83}SeO^{2-}_{4} ->[\beta-][22.3min] ^{83}BrO^-_4 + \beta-}
    \]

    \section{Magic Numbers}
    A magic number is a number of nucleons (either protons or neutrons), such that they
    are arranged into shells within the nucleus. As a result, nuclei with a magic number of
    protons or neutrons are much more stable. Below is the most widely recognized magic numbers.\footnote{As of 2019}
    
    \begin{tabular}{|c|c|}
      \hline
      \textbf{Z Number} & \textbf{Corresponding Element} \\
      \hline
      2   & Helium  \\
      8   & Oxygen  \\
      20  & Calcium \\
      28  & Nickel  \\
      50  & Tin     \\
      82  & Lead    \\
      126 & Unbihexium \\
      \hline
    \end{tabular}\\


    Unlike $2-126$ (which are realized in spherical nuclei), theoretical calculations
    predict that nuclei in the island of stability are deformed. Before people knew this,
    higher magic numbers ($184, 258, 350, \ldots$) were predicted based on a formula
    that assumed spherical spheres:
    \[
      2\left(\begin{pmatrix}n\\ 1\end{pmatrix} + \begin{pmatrix}n\\ 2\end{pmatrix} + \begin{pmatrix}n\\ 3\end{pmatrix}\right)
    \]
    
    It is now known that the sequence of spherical magic numbers cannot be extended in this
    way. Modern calculations predict $228$, and $308$ for neutrons.

    \section{Critical Mass}
    Critical mass is the smallest amount of fissile material for a chain reaction to
    be sustained. This mass depends on the properties of the nuclide. In specific, this
    refers to the nuclear fission cross-section.

    \subsection{Explanation}
    When a nuclear chain reaction is self-sustaining, said mass is said to be \textit{critical}.
    When it is in this critical state there is \textbf{no increase or decrease} in energy. The
    numerical measure of critical mass is dependent on something known as the \textbf{effective neutron multilpication factor}, k.\\

    When $k < 1$, this is known as \textit{subcritical}. These masses are not able to sustain
    a chain reaction. However, when $k = 1$, \textbf{the mass is critical}. Finally, when $k > 1$,
    this is known as supercritical. Supercriticality is when, once fission begins, \textbf{will proceed at an increasing rate}.\\

    It should be noted that supercritical materials are able to go back to criticality,
    whether that is via destroying itself or elevated temperature.

    \subsection{Effective Neutron Multiplication Factor}
    Put it simply, the Effective Neutron Multiplication Factor (ENMF), denoted by $k$,
    is the average number of neutrons fromone fission that will cause another fission. In
    a reactor, k will oscillate from $k < 1$ to $k > 1$, which makes the average $k=1$.

    \subsubsection{Six-Factor Formula}
    $k$ can be defined in only 6 terms. The equation is:
    \begin{equation*}
      \begin{split}
      k = P_\mathrm{FNL}\varepsilon pP_\mathrm{TNL} f\eta \\
      0 < P_\mathrm{FNL} < 1 \\
      1 < \varepsilon < \infty \\
      \end{split}
    \end{equation*}
    
    The factors in this equation are as such:
    \begin{enumerate}
      \item $P_\mathrm{FNL}$: This describes a probability of a
      fast neutron will not escape the system without interacting. If
      this is set to $1$ then fast neutrons will never escape without interacting,
      thus creating an infinite system.
      \item $\varepsilon$: Ratio of total fissions caused \textbf{only} by
      thermal neutrons. If $\varepsilon=1$, it describes a system for which only
      thermal neutrons are causing fission.
      \item $p$: Ratio of the number of neutrons that begin thermalization to
      the number of neutrons that reach thermal energies. Bounds are $0 < p < 1$.
      $1$ defines a system in which no neutrons leak.
      \item $P_\mathrm{TNL}$: Probability that a thermal neutron will not escape
      the system without interacting. Bounds and definition for this factor is the same
      as $P_\mathrm{FNL}$.
      \item $f$: Ratio of number of thermal neutrons absorbed by fissile nuclei vs. the
      number of neutrons absorbed in all materials in the system. Bounds are $0 < f < 1$.
      $1$ denotes a system where everything is fissile nuclei. $0.5$ denotes a system that
      reactions with fissile and non-nuclei are equal.
      \item $\eta$ describes the probability that a neutron absorbed will cause a fission reaction.
    \end{enumerate}

    If the medium is infinite, however, you can set $P_\mathrm{FNL} = P_\mathrm{TNL} = 1$. Another
    thing to note is that $P_\mathrm{FNL}$ is also written as $L_f$, and $P_\mathrm{TNL}$ is written
    as $L_{th}$.

    \subsection{Changing point of criticality}
    There are various ways to perform this. They are shown in subsubsections below.

    \subsubsection*{Varying amount of fuel}
    \subsubsection*{Changing shape}
    \subsubsection*{Changing temperature}
    \subsubsection*{Varying density of mass}
    \subsubsection*{Use of a neutron reflector}
    \subsubsection*{Use of a tamper}

    \subsection{Critical Mass Lookup Table}
    Below is a table of fissile materials, and their critical mass as a bare sphere.\\
    \begin{tabular}{|c|c|c|}
      \hline
      Nuclide & Critical (kg) & Diameter (cm) \\
      \hline
      U-233  & 15  & 11  \\
      U-235  & 52  & 17  \\
      Np-236 & 7   & 8.7 \\
      Np-237 & 60  & 18  \\
      Pu-238 & 9.04& 9.7 \\
      Pu-239 & 10  & 9.9 \\
      Pu-240 & 40  & 15  \\
      Am-241 & 66  & 21.5\\
      Cf-252 & 2.73& 6.9 \\
      Es-254 & 9.89& 7.1 \\
      \hline
    \end{tabular}

    

  \end{multicols*}
\end{document}
