% chktex-file 44
% chktex-file 8
% chktex-file 3

\documentclass{article}

\usepackage{mhchem}
\usepackage{multicol}
\usepackage[margin=50px]{geometry}
\usepackage{enumitem}
\usepackage{titlesec}

\setlist[enumerate]{label*=\arabic*.}


\usepackage{hyperref}
\hypersetup{
  colorlinks=true,
  urlcolor=blue,
  pdftitle={Neutron Science},
}

\title{Neutron Science}
\author{}
\date{2023}

\begin{document}
  \maketitle
  \begin{multicols*}{2}
    \section{Neutron Temperature}
    The neutron temperature, also known as neutron detection temperature,
    indicates a free neuron's kinetic energy, usually given in electron volts.\\
    
    \begin{tabular}{|c|c|}
      \hline
      \textbf{Neutron Energy} & \textbf{Energy Range} \\
      \hline
      $0.0 - 0.025$ eV & Cold (slow) Neutrons \\
      $0.025$ eV & Thermal neutrons ($20$ C) \\
      $0.025-0.4$ eV & Epithermal neutrons \\
      $0.4-0.5$ eV & Cadmium neutrons \\
      $0.5-10$ eV & Epicadmium neutrons \\
      $10-300$ eV & Resonance neutrons \\
      $300$ eV $- 1$ MeV & Intermediate neutrons \\
      $1-20$ MeV & Fast neutrons \\
      $> 20$ MeV & Ultrafast neutrons \\
      \hline
    \end{tabular}


    \section{Neutron Flux and Neutron Fluence}
    \subsection{Neutron Flux}
    \subsection{Neutron Fluence}
  \end{multicols*}
\end{document}