% chktex-file 44
% chktex-file 8
% chktex-file 3

\documentclass{article}

\usepackage{mhchem}
\usepackage{multicol}
\usepackage[margin=50px]{geometry}
\usepackage{enumitem}
\usepackage{titlesec}

\setlist[enumerate]{label*=\arabic*.}


\usepackage{hyperref}
\hypersetup{
  colorlinks=true,
  urlcolor=blue,
  pdftitle={Neutron Science},
}

\title{Neutron Science}
\author{}
\date{2023}

\begin{document}
  \maketitle
  \begin{multicols*}{2}
    \section{Neutron Temperature}
    The neutron temperature, also known as neutron detection temperature,
    indicates a free neuron's kinetic energy, usually given in electron volts.\\
    
    \begin{tabular}{|c|c|}
      \hline
      \textbf{Neutron Energy} & \textbf{Energy Range} \\
      \hline
      $0.0 - 0.025$ eV & Cold (slow) Neutrons \\
      $0.025$ eV & Thermal neutrons ($20$ C) \\
      $0.025-0.4$ eV & Epithermal neutrons \\
      $0.4-0.5$ eV & Cadmium neutrons \\
      $0.5-10$ eV & Epicadmium neutrons \\
      $10-300$ eV & Resonance neutrons \\
      $300$ eV $- 1$ MeV & Intermediate neutrons \\
      $1-20$ MeV & Fast neutrons \\
      $> 20$ MeV & Ultrafast neutrons \\
      \hline
    \end{tabular}\\

    To convert eV to Joules, for use in neutron flux, you just
    divide it by $6.242 \times 10^{18}$:
    \[
      E_J = \frac{E_e}{6.242\times10^{18}}
    \]

    For MeV, you just divide by $6.242\times10^{12}$.


    \section{Neutron Flux and Neutron Fluence}
    \subsection{Neutron Flux}
    Neutron flux is a measure of the intensity of neutron radiation,
    which is determined by the rate of flow of neutrons. It is calculated
    via
    \[
      \varphi = nv,
    \] 

    where $n$ is the neutron density, and $v$ is the distance neutrons travel
    in one second. $n$ is approximately $1.78\cdot 10^{53}$ To find $v$, you can do,
      \[
        v = \frac{v_1}{v_2}
      \]\[
        v_1 = \sqrt{\frac{2E}{m}},\;\;\;
        v_2 = 1 \cdot \left(1 \cdot 10^{15}\right),
      \]

      where $E$ is the energy of the neutron in Joules and $m$ is the mass of a neutron.

    \subsection{Neutron Fluence}
    Neutron fluence is just neutron flux multiplied by time $t$:
    \[
      L = \varphi{t}.
    \]

    $t$ is the duration of your flux `experiment', in seconds.

    \section{Neutron Cross Section}
    Neutron cross section is a unit used to express the \textbf{likelihood} that
    a target neutron will impact a neutron. The reaction between a nucleus and a
    neutron is
    \[
      \ce{^{A}_{Z}X + ^{1}_{0}n -> ^{A+1}_{Z}X}
    \]

    
    I'm gonna put something here later, when I figure out how to do this.
  
    \section{Neutron Sources}
    Neutron sources are an easy way to generate neutrons in the lab. Put
    it simply, they generate neutrons (hence the name, neutron source).

    \subsection{How do they work?}
    In the simplest neutron source, it's a mix of an $\alpha$-emitter and
    a lighter isotope, like boron and carbon. For our intents, assume we
    have a neutron source composed of $\ce{^{226}_{88}Ra}$ and $\ce{^{12}_{6}C}$.\\

    When the $\ce{^{226}Ra}$ decays, it emits an $\alpha$ particle. When this
    $\alpha$ particle comes into contact with the carbon, it emits a neutron,
    and the carbon turns into oxygen. The reaction is characterized as such:

    \begin{equation*}
      \begin{split}
        \ce{^{226}_{88}Ra -> ^{222}_{86}Rn + ^{4}_{2}He} \\
        \ce{^{12}_{6}C + ^{4}_{2}He -> ^{15}_{8}O + \boxed{^{1}_{0}\ce{n}}}
      \end{split}
    \end{equation*}
    \subsection{Determining neutron source efficiency}
  \end{multicols*}
\end{document}