% chktex-file 44
% chktex-file 8
% chktex-file 3

\documentclass{article}

\usepackage{mhchem}
\usepackage{multicol}
\usepackage[margin=50px]{geometry}
\usepackage{enumitem}
\usepackage{titlesec}

\setlist[enumerate]{label*=\arabic*.}


\usepackage{hyperref}
\hypersetup{
  colorlinks=true,
  urlcolor=blue,
  pdftitle={Neutron Science},
}

\title{Neutron Science}
\author{}
\date{2023}

\begin{document}
  \maketitle
  \begin{multicols*}{2}
    \section{Neutron Temperature}
    The neutron temperature, also known as neutron detection temperature,
    indicates a free neuron's kinetic energy, usually given in electron volts.\\
    
    \begin{tabular}{|c|c|}
      \hline
      \textbf{Neutron Energy} & \textbf{Energy Range} \\
      \hline
      $0.0 - 0.025$ eV & Cold (slow) Neutrons \\
      $0.025$ eV & Thermal neutrons ($20$ C) \\
      $0.025-0.4$ eV & Epithermal neutrons \\
      $0.4-0.5$ eV & Cadmium neutrons \\
      $0.5-10$ eV & Epicadmium neutrons \\
      $10-300$ eV & Resonance neutrons \\
      $300$ eV $- 1$ MeV & Intermediate neutrons \\
      $1-20$ MeV & Fast neutrons \\
      $> 20$ MeV & Ultrafast neutrons \\
      \hline
    \end{tabular}
    \section{Neutron Flux}
    Neutron flux, denoted by $\phi$, is a quantity that is equal to the total
    distance travelled by all free neutrons per unit time and volume. The usual
    unit for measuring neutron flux is cm$^{-2}$s$^{-1}$.
    \subsection{Natural}
    Neutron flux in AGB\footnote{asymptotic giant branch} stars and in supernovae
    is responsible for nucleosynthesis that produces elements heavier then iron.
    In stars, the neutron flux is slow, at $10^{5} - 10^{11}$ cm$^{-2}$s$^{-1}$.
    This results in nucleosynthesis by the s-process.

    In constrast, however, in supernovae, the neutron flux is extremely high,
    on the order of $10^{32}$ cm$^{-2}$s$^{-1}$. This reults in nucleosynthesis
    via the r-process.
    \subsection{Artifical}
    This categorization of neutron flux refers which is man-made, either the
    byproducts from some nuclear weapon or nuclear energy production or for
    some specific application.\\

    A flow of neutrons from a neutron source is used to initiate a fission reaction
    of unstable large nuclei. These additional neutrons may cause the nuclei to
    become unstable, and split into more stable products. This effect is used
    in fission reactors and nuclear weapons.
    \section{Neutron cross section}
    A neutron cross section, denoted by $\sigma$, is used to express the likelihood
    of interaction between a neutron and a target nucleus. The standard unit for
    measuring the cross-section is called a \textbf{barn}, which is equal to
    $10^{-24}$ cm$^{2}$.

    \subsection{Parameters of Interest}
    The neutron cros section depends on a few parameters:
    \begin{itemize}
      \item The target type (hydrogen, uranium, \ldots)
      \item The type of nuclear reaction
      \item Speed of particle energy (thermal, fast, \ldots)
    \end{itemize}
    \subsection{Calculating}
    The equation to calculate a neutron cross-section is
    \[
      \sigma = \sigma_{0}\left( \frac{T_0}{T} \right)^{0.5}
    \]

    Where $\sigma$ is the cross section at temperature $T$, and $\sigma_0$ is
    the cross section at temperature $T_0$. Both $T$ and $T_0$ are in Kelvin.

    \subsubsection{`Okay, how do we get $\sigma_0$?'}
    
  \end{multicols*}
\end{document}