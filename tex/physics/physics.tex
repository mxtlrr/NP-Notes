% chktex-file 44
% chktex-file 8
% chktex-file 3

\documentclass{article}

\usepackage{mhchem}
\usepackage{multicol}
\usepackage[margin=50px]{geometry}
\usepackage{enumitem}
\usepackage{titlesec}

\setlist[enumerate]{label*=\arabic*.}


\usepackage{hyperref}
\hypersetup{
  colorlinks=true,
  urlcolor=blue,
  pdftitle={Neutron Science},
}

\title{Neutron Science}
\author{}
\date{2023}

\begin{document}
  \maketitle
  \begin{multicols*}{2}
    \section{Neutron Temperature}
    The neutron temperature, also known as neutron detection temperature,
    indicates a free neuron's kinetic energy, usually given in electron volts.\\
    
    \begin{tabular}{|c|c|}
      \hline
      \textbf{Neutron Energy} & \textbf{Energy Range} \\
      \hline
      $0.0 - 0.025$ eV & Cold (slow) Neutrons \\
      $0.025$ eV & Thermal neutrons ($20$ C) \\
      $0.025-0.4$ eV & Epithermal neutrons \\
      $0.4-0.5$ eV & Cadmium neutrons \\
      $0.5-10$ eV & Epicadmium neutrons \\
      $10-300$ eV & Resonance neutrons \\
      $300$ eV $- 1$ MeV & Intermediate neutrons \\
      $1-20$ MeV & Fast neutrons \\
      $> 20$ MeV & Ultrafast neutrons \\
      \hline
    \end{tabular}\\

    To convert eV to Joules, for use in neutron flux (see below), you just
    divide it by $6.242 \times 10^{18}$:
    \[
      E_J = \frac{E_e}{6.242\times10^{18}}
    \]

    For MeV, you just divide by $6.242\times10^{12}$.


    \section{Neutron Flux and Neutron Fluence}
    \subsection{Neutron Flux}
    Neutron flux is a measure of the intensity of neutron radiation,
    which is determined by the rate of flow of neutrons. It is calculated
    via
    \[
      \varphi = nv,
    \] 

    where $n$ is the neutron density, and $v$ is the distance neutrons travel
    in one second. $n$ is approximately $1.78\cdot 10^{53}$ To find $v$, you can do,
      \[
        v = \frac{v_1}{v_2}
      \]\[
        v_1 = \sqrt{\frac{2E}{m}},\;\;\;
        v_2 = 1 \cdot \left(1 \cdot 10^{15}\right),
      \]

      where $E$ is the energy of the neutron in Joules and $m$ is the mass of a neutron.

    \subsection{Neutron Fluence}
    Neutron fluence is just neutron flux multiplied by time $t$:
    \[
      L = \varphi{t}.
    \]

    $t$ is the duration of your flux `experiment', in seconds.

    \section{Neutron Cross Section}
    Neutron cross section is a unit used to express the \textbf{likelihood} that
    a target neutron will impact a neutron. The reaction between a nucleus and a
    neutron is
    \[
      \ce{^{A}_{Z}X + ^{1}_{0}n -> ^{A+1}_{Z}X}
    \]

    The equation to calculate neutron cross-section is as follows:
    \[
      \sigma_t = \frac{C}{N_a \times I_0},
    \]

    where $C$ is the number of interactions, $N_a$ is atom density, and finally,
    $I_0$ is neutron fluence (intensity). To calculate $N_a$, you use this formula:
    \[
      N_a = \frac{D\times(6.02\cdot10^{23})}{M},
    \]

    where $D$ is the density of the radioisotope, and $M$ is the mass (in Da) of
    said radionuclide.

    \subsection{Example -- Copper-63}
    The first step we want to do is figure out neutron flux, then do neutron fluence,
    and finally calculate neutron cross section. For this example, let's say we're only
    dealing with 200 interactions.

    \subsubsection{Neutron Fluence}
    First we need to figure out $\varphi$ (neutron flux),
    \begin{equation*}
      \begin{split}
        \varphi = nv = 1.78\times10^{53} \cdot v\\
        v = \frac{\sqrt{\frac{2\left(4.005\times10^{-21}\right)}{1.674\times10^{-27}}}}{1\cdot \left(1\times10^{15}\right)}\\
        = \frac{2187.45}{1\times10^{15}}\\
        = 2.187 \times 10^{-12}\\
        \varphi = \left(1.78\times10^{53}\right) \cdot \left(2.18 \times 10^{-12}\right)\\
        = \boxed{3.89 \times 10^{41}}.
      \end{split}
    \end{equation*}

    The next step is to multiply it by the duration of our experiment, which is going to
    be 0.1 seconds as an example:
    \begin{equation*}
      \begin{split}
      L = 0.1\varphi\\
      = 0.1 \times 3.89\times10^{41}\\
      = 3.80\times10^{40}.
      \end{split}
    \end{equation*}

    Now for the fun part. Since we have the neutron fluence, we can calculate the neutron
    cross-section.

    \begin{equation*}
      \begin{split}
        N_a = \frac{7.93 \times \left(6.02\cdot 10^{23}\right)}{62.9}\\
        = \frac{4.773\times 10^{24}}{62.9}\\
        = \boxed{7.58\times10^{22}}\\\\
        \sigma = \frac{200}{\left(7.58 \times 10^{22}\right) \times \left(3.89\times10^{40}\right)}\\
        = \frac{200}{2.949\times10^{63}}\\
        = \boxed{6.78 \times 10^{-62}}
      \end{split}
    \end{equation*}
    Thus, the neutron cross section for 200 thermal neutrons for $\ce{^{63}_{29}Cu}$ is
    $6.78 \times 10^{-62}\;$ barns.
  \end{multicols*}
\end{document}